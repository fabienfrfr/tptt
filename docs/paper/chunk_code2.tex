
% 代码样式设置
\lstset{
    language=Python,
    basicstyle=\ttfamily\small,
    numbers=left,
    numberstyle=\ttfamily\textcolor[rgb]{0.5,0.5,0.5}{\scriptsize},
    numbersep=8pt,
    frame=lines,
    framesep=2mm,
    breaklines=true,
    keywordstyle=\color{blue},
    commentstyle=\color[rgb]{0,0.6,0}{\itshape},
    stringstyle=\color{red},
    showstringspaces=false,
    columns=flexible,
    mathescape=true
}

% 代码部分
\begin{figure}[h]
\centering
\begin{lstlisting}
def chunk_delta_rule_forward(Q, K, V, beta, C):
    '''
    Q/K/V: query, key, value of shape [L, d]
    beta: beta of shape [L]
    C: chunk size 
    '''
    # L: sequence length, d: head dimension 
    L, d = Q.shape
    # chunking
    Q, K, V = map(lambda x: x.reshape(-1,C,d), [Q, K, V])
    beta = beta.reshape(-1, C)
    K_beta = K * beta.unsqueeze(-1)
    V_beta = V * beta.unsqueeze(-1)
    
    # compute eq. 10 with vectorized forword substitution for fast inverse
    T = -(K_beta @ K.t()).tril(-1)
    for i in range(1, C):
        T[i, :i] = T[i, :i] + (T[i, :, None] * T[:, :i]).sum(-2)
    T += torch.eye(C)
    # compute Eq. 11
    W = T @ K_beta
    U = T @ V_beta
    # chunkwise parallel. Eq. 8-9
    S = torch.zeros(d, d)
    O = torch.empty_like(V)
    for i in range(L//C):
        q_i, k_i, w_i = Q[i], K[i], W[i]
        u_i = U[i] - w_i @ S
        o_inter = q_i @ S
        A_i = (q_i @ k_i.t()).tril()
        o_intra = A_i @ u_i
        S += k_i.t() @ u_i
        O[i] = o_intra + o_inter
    return O.reshape(L, d)
\end{lstlisting}
\caption{Pytorch-like code snippet of the forward pass of our chunkwise algorithm for training DeltaNet. We omit the dimensions of batch size and number of heads for clarity.}
\label{list_code}
\end{figure}